\documentclass[11pt,a4paper]{article}
\usepackage{xcolor}
\usepackage{CJKutf8}
\usepackage{graphicx}
\usepackage{amsmath}
\usepackage{braket}
\usepackage{caption}
\usepackage{bm}
\usepackage{geometry}
\geometry{a4paper,scale=0.8}

\captionsetup{font={scriptsize}}

\begin{CJK}{UTF8}{gbsn}
\begin{document}

\title{NEGF Notes}
% \author{Li Gaoyang}
% \date{}
\date{\today}
\maketitle
% \tableofcontents

% \newpage

\subsection{Hamiltonian}
\begin{equation}
H=H_{L}+H_{R}+H_{d}+H_{T}+H_{s d}
\end{equation}
\begin{equation}
H_{L}=\sum_{k \sigma} \epsilon_{k \sigma, L} c_{k \sigma}^{\dagger} c_{k \sigma}
\end{equation}
\begin{equation}
H_{R}=\sum_{q} \omega_{q} a_{q}^{\dagger} a_{q}
\end{equation}
\begin{equation}
H_{d}=\sum_{n \sigma} \epsilon_{n \sigma} d_{n \sigma}^{\dagger} d_{n \sigma}
\end{equation}
\begin{equation}
H_{T}=\sum_{k \sigma n}\left(t_{k \sigma n} c_{k \sigma}^{\dagger} d_{n \sigma}+t_{k \sigma n}^{*} d_{n \sigma}^{\dagger} c_{k \sigma}\right)
\end{equation}
\begin{equation}
H_{s d}=-\sum_{q n m} J_{q}\left(d_{n \uparrow}^{\dagger} d_{m \downarrow} a_{q}^{\dagger}+a_{q} d_{m \downarrow}^{\dagger} d_{n \uparrow}\right) \delta\left(\epsilon_{n \uparrow}-\epsilon_{m \downarrow}-\omega_{q}\right)
\end{equation}
\begin{equation}
s_{q}^{+}=\sum_{n m} d_{n \uparrow}^{\dagger} d_{m \downarrow} \delta_{\uparrow \downarrow}
\end{equation}
\begin{equation}
s_{q}^{-}=\sum_{n m} d_{m \downarrow}^{\dagger} d_{n \uparrow} \delta_{\uparrow \downarrow}
\end{equation}
\subsubsection{check operators}
\begin{equation}
i \dot{a}_{q}=\omega_{q} a_{q}-J_{q} s_{q}^{+}
\end{equation}
\begin{equation}
i \dot{c}_{k \sigma}=\epsilon_{k \sigma, L} c_{k \sigma}+\sum_{k^{\prime}} t_{k \sigma n} d_{n \sigma}
\end{equation}
\begin{equation}
i \dot{d}_{n \uparrow}=\epsilon_{n \uparrow} d_{n \uparrow}+\sum_{k} t_{k \uparrow n}^{*} c_{k \uparrow}-\sum_{q, m} J_{q} a_{q}^{\dagger}d_{m \downarrow}\delta(\epsilon_{n \uparrow}-\epsilon_{m \downarrow}-\omega_{q})
\end{equation}
\begin{equation}
i \dot{d}_{n \downarrow}=\epsilon_{n \downarrow} d_{n \downarrow}+\sum_{k} t_{k \downarrow n}^{*} c_{k \downarrow}-\sum_{q, m} J_{q} a_{q}d_{m \uparrow}\delta(\epsilon_{m \uparrow}-\epsilon_{n \downarrow}-\omega_{q})
\end{equation}
\subsection{spin current ???}
Define
\begin{equation}
G_{d, R}\left(\tau, \tau^{\prime}\right)=-i\langle s_{q}^{+}(\tau) a_{q}^{\dag}(\tau')\rangle.
\end{equation}
The lesser Green's function is($s_{q}^{+}$ is fermionic but $a_{q}$ is bosonic)
\begin{equation}
G_{d, R}^{<}\left(t, t'\right)=-i\langle  a_{q}^{\dag}(t') s_{q}^{+}(t) \rangle
\end{equation}
We also define the Green’s function that is related to the QD (not the Green’s function of the QD),
\begin{equation}
G_{d}\left(\tau, \tau^{\prime}\right)= -i\left\langle T_{c} S s_{q}^{+}(\tau) s_{q}^{-}\left(\tau^{\prime}\right)\right\rangle .
\end{equation}
We have
\begin{equation}
-i \partial_{\tau^{\prime}} G_{d, R}\left(\tau, \tau^{\prime}\right)=\omega_{q} G_{d, R}\left(\tau, \tau^{\prime}\right)-J_{q} G_{d}
\end{equation}
or
\begin{equation}
G_{d, R} g_{R q}^{-1}=-J_{q} G_{d}
\end{equation}
or
\begin{equation}
G_{d, R}\left(\tau, \tau^{\prime}\right)=-J_{q} \int G_{d}\left(\tau, \tau_{1}\right) g_{R q}\left(\tau_{1}, \tau^{\prime}\right) d \tau_{1}
\end{equation}
the minus before $J_{q}$ originates from the minus in $H_{sd}$. The rules of analytic continuation gives
\begin{equation}
G_{d, R}^{<}\left(t, t^{\prime}\right)=-J_{q} \int_{-\infty}^{\infty}dt_{1} [G_{d}^{r}\left(t, t_{1}\right) g_{R q}^{<}\left(t_{1}, t^{\prime}\right) + G_{d}^{<}\left(t, t_{1}\right) g_{R q}^{a}\left(t_{1}, t^{\prime}\right)]
\end{equation}
and
\begin{equation}
G_{R,d}^{<}\left(t, t^{\prime}\right)=-J_{q} \int_{-\infty}^{\infty}dt_{1} [g_{R q}^{r}\left(t, t_{1}\right) G_{d}^{<}\left(t_{1}, t'\right)  + g_{R q}^{<}\left(t, t_{1}\right) G_{d}^{a}\left(t_{1}, t'\right)]
\end{equation}
The spin current flows out of right lead is
\begin{equation}
\begin{split}
I_{s}&= i \sum_{q} J_{q}\left(\left\langle s_{q}^{+} a_{q}^{\dagger}\right\rangle-\left\langle a_{q} s_{q}^{-}\right\rangle\right) \\
&=-\sum_{q}J_{q}(G_{d,R}^{<}(t,t)-G_{R,d}^{<}(t,t)) \\
&=2\rm{Re}\sum_{q} \int  d t_{1}\operatorname{Tr}\left[G_{d}^{r}\left(t, t_{1}\right) \Sigma_{Rq}^{<}\left(t_{1}, t^{\prime}\right)+G_{d}^{<}\left(t, t_{1}\right) \Sigma_{Rq}^{a}\left(t_{1}, t^{\prime}\right)\right]
% \\ &=2\rm{Re}\sum_{q} \int  d t_{1}\operatorname{Tr}\left[\left(G_{d}^{>}-G_{d}^{<}\right) \Sigma_{R q}^{<}+G_{d}^{<}\left(\Sigma_{R q}^{a}-\Sigma_{R q}^{r}\right)\right]
\end{split}
\end{equation}
\begin{equation}
\Sigma_{Rq}^{\gamma}(\tau, \tau')=J_{q}^{2}g_{Rq}^{\gamma}(\tau, \tau')
\end{equation}
\subsection{Calculation of $G_{d}$}
Definition:
\begin{equation}
\begin{split}
G_{d}\left(\tau, \tau^{\prime}\right)&=
-i\left\langle T_{c} S s_{q}^{+}(\tau) s_{q}^{-}\left(\tau^{\prime}\right)\right\rangle \\
&=-i\sum_{mnm'n'}\langle T_{c}S d_{n \uparrow}^{\dagger} d_{m \downarrow} d_{m' \downarrow}^{\dagger} d_{n' \uparrow} \rangle \delta\left(\epsilon_{n \uparrow}-\epsilon_{m \downarrow}-\omega_{q}\right) \delta\left(\epsilon_{n' \uparrow}-\epsilon_{m' \downarrow}-\omega_{q}\right) 
\end{split}
\end{equation}
When right lead is absent, the system Hamiltonian is
\begin{equation}
H = H_{L} + H_{d} + H_{T}.
\end{equation}
\begin{equation}
G_{d}\left(\tau, \tau^{\prime}\right)=-i\sum_{mnm'n'}G_{L,n'n\uparrow}(\tau', \tau) G_{L,mm'\downarrow}(\tau, \tau') \delta\left(\epsilon_{n \uparrow}-\epsilon_{m \downarrow}-\omega_{q}\right) \delta\left(\epsilon_{n' \uparrow}-\epsilon_{m' \downarrow}-\omega_{q}\right)
\end{equation}
where
\begin{equation}
\begin{split}
G_{L,mn\sigma}(\tau, \tau')&= -i\langle T_{c}d_{m\sigma}(\tau) d_{n\sigma}^{\dag}(\tau')\rangle \\
&=g_{mn\sigma}(\tau, \tau')\delta_{mn} \\
&~~~+ \iint d\tau_{1}d\tau_{2} g_{mm\sigma}(\tau, \tau_{2}) \sum_{k} t_{k\sigma n}t_{k\sigma m}^{*} g_{k\sigma}(\tau_{2}, \tau_{1}) g_{nn\sigma}(\tau_{1}, \tau') \\
&~~~ +\cdots \\
&=g_{mn\sigma}(\tau, \tau')\delta_{mn} + \iint d\tau_{1}d\tau_{2} g_{mm\sigma}(\tau, \tau_{2}) \Sigma_{L,mn\sigma}(\tau_{2}, \tau_{1}) g_{nn\sigma}(\tau_{1}, \tau')\\
&~~~+ \cdots \\
&= 1 /\left[g_{mn \sigma}^{-1}-\Sigma_{L,mn \sigma}\right]
\end{split}
\end{equation}
\begin{equation}
g_{mn\sigma}(\tau, \tau') = -i\langle T_{c}d_{m\sigma}(\tau) d_{n\sigma}^{\dag}(\tau')\rangle_{0}
\end{equation}
Self-energy of left lead
\begin{equation}
\Sigma_{L,mn\sigma}(\tau_{2}, \tau_{1}) = \sum_{k} t_{k\sigma n}t_{k\sigma m}^{*} g_{k\sigma}(\tau_{2}, \tau_{1})
\end{equation}
where
\begin{equation}
g_{k\sigma}(\tau_{2}, \tau_{1}) = -i\langle T_{c}c_{k\sigma}(\tau_{2}) c_{k\sigma}^{\dag}(\tau_{1})\rangle_{0}.
\label{eq:left-self-energy}
\end{equation}
When left lead is absent, system Hamiltonian is
\begin{equation}
H = H_{d} + H_{R} + H_{sd}.
\end{equation}
\begin{equation}
\begin{split}
G_{d}\left(\tau, \tau^{\prime}\right) =&-i\sum_{mn} g_{n \uparrow}\left(\tau^{\prime}, \tau\right) g_{m \downarrow}\left(\tau, \tau^{\prime}\right) \delta(\varepsilon_{n\uparrow} - \varepsilon_{m\downarrow} - \omega_{q})\\
&-\int d\tau_{1}\int d\tau_{2} \sum_{mnm'n'}g_{n \uparrow}\left(\tau_{1}, \tau\right) g_{m \downarrow}\left(\tau, \tau_{1}\right) \Sigma_{R,mnm'n' }\left(\tau_{1}, \tau_{2}\right) g_{n'\uparrow}\left(\tau^{\prime}, \tau_{2}\right)g_{m'\downarrow}(\tau_{2}, \tau') \\
&~~~\times \delta(\varepsilon_{n\uparrow} - \varepsilon_{m\downarrow} - \omega_{q_{1}})\delta(\varepsilon_{n'\uparrow} - \varepsilon_{m'\downarrow} - \omega_{q_{1}})\\
& + \cdots \\
=& g_{d}(\tau, \tau') + \iint d\tau_{1}d\tau_{2}g_{d}(\tau, \tau_{1}) \Sigma_{R}(\tau_{1}, \tau_{2}) G_{d}(\tau_{2}, \tau')
\end{split}
\end{equation}
in which, 
\begin{equation}
g_{d}\left(\tau, \tau^{\prime}\right) = -i\sum_{mn} g_{n \uparrow}\left(\tau^{\prime}, \tau\right) g_{m \downarrow}\left(\tau, \tau^{\prime}\right) \delta(\varepsilon_{n\uparrow} - \varepsilon_{m\downarrow} - \omega_{q}),
\end{equation}
the self-energy of right lead is
\begin{equation}
\Sigma_{R,mnm'n'}(\tau_{1}, \tau_{2}) = \sum_{q1}J_{q_{1}}^{2} g_{Rq_{1}}(\tau_{1}, \tau_{2}) \delta(\varepsilon_{n\uparrow} - \varepsilon_{m\downarrow} - \omega_{q_{1}})\delta(\varepsilon_{n'\uparrow} - \varepsilon_{m'\downarrow} - \omega_{q_{1}})
\end{equation}
\begin{equation}
g_{Rq_{1}}(\tau_{1}, \tau_{2}) = -i\langle T_{c}a_{q_{1}}(\tau_{1}) a_{q_{1}}^{\dag}(\tau_{2})\rangle_{0}
\end{equation}
Hence, when both leads are present, we have
\begin{equation}
\begin{split}
G_{d}\left(\tau, \tau^{\prime}\right)=&-i \sum_{mnm'n'}G_{L,nn' \uparrow}\left(\tau^{\prime}, \tau\right) G_{L,mm' \downarrow}\left(\tau, \tau^{\prime}\right)\delta\left(\epsilon_{n \uparrow}-\epsilon_{m \downarrow}-\omega_{q}\right) \delta\left(\epsilon_{n' \uparrow}-\epsilon_{m' \downarrow}-\omega_{q}\right)\\
&-i \sum_{mnm'n'}G_{L,nn' \uparrow}\left(\tau_{1}, \tau\right) G_{L,mm' \downarrow}\left(\tau, \tau_{1}\right) \Sigma_{R,mnm'n'}\left(\tau_{1}, \tau_{2}\right) G_{d}\left(\tau_{2}, \tau^{\prime}\right) \delta\left(\epsilon_{n \uparrow}-\epsilon_{m \downarrow}-\omega_{q}\right)\\
&\qquad\times \delta\left(\epsilon_{n' \uparrow}-\epsilon_{m' \downarrow}-\omega_{q}\right)
\label{eq:1}
\end{split}
\end{equation}
For the sack of convenience, we rewrite the above formula in matrix presentation as follows(the matrix indices are QD level indices $m,n$, not corrected yet!), and omit energy conservation constrain.
\begin{equation}
{\color{red}{?}}~~G_{d}\left(\tau, \tau^{\prime}\right)=-i G_{L \uparrow}\left(\tau^{\prime}, \tau\right) G_{L \downarrow}\left(\tau, \tau^{\prime}\right)-i G_{L \uparrow}\left(\tau_{1}, \tau\right) G_{L \downarrow}\left(\tau, \tau_{1}\right) \Sigma_{R }\left(\tau_{1}, \tau_{2}\right) G_{d}\left(\tau_{2}, \tau^{\prime}\right)
\end{equation}
\subsection{continuation on Eq.(\ref{eq:1})}
\begin{eqnarray}
% \center
A(\tau_{1}, \tau') \equiv\int d\tau_{2}\Sigma_{R }\left(\tau_{1}, \tau_{2}\right) G_{d}\left(\tau_{2}, \tau^{\prime}\right)\\
B(\tau, \tau_{1}) \equiv G_{L \uparrow}\left(\tau_{1}, \tau\right) G_{L \downarrow}\left(\tau, \tau_{1}\right)\\
C(\tau, \tau')\equiv -i G_{L \uparrow}\left(\tau_{1}, \tau\right) G_{L \downarrow}\left(\tau, \tau_{1}\right) A(\tau_{1}, \tau') \rightarrow \\
C(\tau, \tau') = -i\int d\tau_{1}B(\tau, \tau_{1})A(\tau_{1}, \tau')\\
D(\tau, \tau') \equiv -i G_{L \uparrow}\left(\tau^{\prime}, \tau\right) G_{L \downarrow}\left(\tau, \tau^{\prime}\right)
\end{eqnarray}
So, we have
\begin{equation}
G_{d}(\tau, \tau') = D + C
\end{equation}
Using the analytic continuation theorem, we have
\begin{equation}
D^{<} = -i G_{L \uparrow}^{>} G_{L \downarrow}^{<}
\end{equation}
\begin{equation}
C^{<}=-i(B^{r} A^{<}+B^{<} A^{a})
\end{equation}
where
\begin{equation}
B^{r} = G_{L \uparrow}^{a} G_{L \downarrow}^{<}+G_{L \uparrow}^{>} G_{L \downarrow}^{r}+G_{L \uparrow}^{a} G_{L \downarrow}^{r}
\end{equation}
\begin{equation}
A^{<}=\Sigma_{R}^{r} G_{d}^{<}+\Sigma_{R}^{<} G_{d}^{a}
\end{equation}
\begin{equation}
B^{<}=G_{L\uparrow}^{>} G_{L\downarrow}^{<}
\end{equation}
\begin{equation}
A^{a}=\Sigma_{R}^{a} G_{d}^{a}
\end{equation}
Then, the analytic continuation theorem on Eq.(\ref{eq:1}) yields
\begin{equation}
G_{d}^{<}=-i G_{L \uparrow}^{>} G_{L \downarrow}^{<} -i\left[(G_{L \uparrow}^{a} G_{L \downarrow}^{<}+G_{L \uparrow}^{>} G_{L \downarrow}^{r}+G_{L \uparrow}^{a} G_{L \downarrow}^{r})(\Sigma_{R}^{r} G_{d}^{<}+\Sigma_{R}^{<} G_{d}^{a}) + (G_{L\uparrow}^{>} G_{L\downarrow}^{<})(\Sigma_{R}^{a} G_{d}^{a})\right]
\label{eq:Gd<}
\end{equation}
Similarly, 
\begin{equation}
\begin{split}
C^{r}&=-iB^{r} A^{r} \\
&=-i(G_{L \uparrow}^{a} G_{L \downarrow}^{<}+G_{L \uparrow}^{>} G_{L \downarrow}^{r}+G_{L \uparrow}^{a} G_{L \downarrow}^{r})(\Sigma_{R}^{r} G_{d}^{r})
\end{split}
\end{equation}
\begin{equation}
D^{r}=-i(G_{L \uparrow}^{a} G_{L \downarrow}^{<}+G_{L \uparrow}^{>} G_{L \downarrow}^{r}+G_{L \uparrow}^{a} G_{L \downarrow}^{r}),
\end{equation}
we have
\begin{equation}
\begin{split}
G_{d}^{r} &=  -i(G_{L \uparrow}^{a} G_{L \downarrow}^{<}+G_{L \uparrow}^{>} G_{L \downarrow}^{r}+G_{L \uparrow}^{a} G_{L \downarrow}^{r})-i(G_{L \uparrow}^{a} G_{L \downarrow}^{<}+G_{L \uparrow}^{>} G_{L \downarrow}^{r}+G_{L \uparrow}^{a} G_{L \downarrow}^{r})(\Sigma_{R}^{r} G_{d}^{r}) \\
&=\frac{-i\left(G_{L \uparrow} G_{L \downarrow}\right)^{r}}{ 1+i\left(G_{L \uparrow} G_{L \downarrow}\right)^{r} \Sigma_{R}^{r}}
\end{split}
\end{equation}
\begin{equation}
(G_{L \uparrow} G_{L \downarrow})^{r} \equiv G_{L \uparrow}^{a} G_{L \downarrow}^{<}+G_{L \uparrow}^{>} G_{L \downarrow}^{r}+G_{L \uparrow}^{a} G_{L \downarrow}^{r}
\end{equation}
Now we calculate $G_{d}^{a}$.
\begin{equation}
C^{a}=-iB^{a} A^{a}
\end{equation}
\begin{equation}
B^{a}=G_{L \uparrow}^{r} G_{L \downarrow}^{<}+G_{L \uparrow}^{>} G_{L \downarrow}^{a}+G_{L \uparrow}^{r} G_{L \downarrow}^{a}
\end{equation}
\begin{equation}
D^{a}=-i(G_{L \uparrow}^{r} G_{L \downarrow}^{<}+G_{L \uparrow}^{>} G_{L \downarrow}^{a}+G_{L \uparrow}^{r} G_{L \downarrow}^{a})
\end{equation}
So we have
\begin{equation}
G_{d}^{a} = -i(G_{L \uparrow}^{r} G_{L \downarrow}^{<}+G_{L \uparrow}^{>} G_{L \downarrow}^{a}+G_{L \uparrow}^{r} G_{L \downarrow}
^{a})-i(G_{L \uparrow}^{r} G_{L \downarrow}^{<}+G_{L \uparrow}^{>} G_{L \downarrow}^{a}+G_{L \uparrow}^{r} G_{L \downarrow}^{a})(\Sigma_{R}^{a} G_{d}^{a})
\end{equation}
From Eq.(\ref{eq:Gd<}) we have
\begin{equation}
\begin{split}
G_{d}^{<}&=-i G_{L \uparrow}^{>} G_{L \downarrow}^{<}\left(1+\Sigma_{R}^{a} G_{d}^{a}\right)-i\left(G_{L \uparrow} G_{L \downarrow}\right)^{r}\left(\Sigma_{q}^{<} G_{d}^{a}+\Sigma_{R}^{r} G_{d}^{<}\right)\\
&=\frac{-i G_{L \uparrow}^{>} G_{L \downarrow}^{<}\left(1+\Sigma_{R}^{a} G_{d}^{a}\right) - i\left(G_{L \uparrow} G_{L \downarrow}\right)^{r}\Sigma_{R}^{<} G_{d}^{a}}{1+i\left(G_{L \uparrow} G_{L \downarrow}\right)^{r}\Sigma_{R}^{r}} \\
&=\frac{-i G_{L \uparrow}^{>} G_{L \downarrow}^{<}\left(1+\Sigma_{R}^{a} G_{d}^{a}\right)}{1+i\left(G_{L \uparrow} G_{L \downarrow}\right)^{r}\Sigma_{R}^{r}} + G_{d}^{r}\Sigma_{R}^{<} G_{d}^{a}\\
&=-i(G_{d}^{r}\Sigma_{R}^{r}+1) G_{L \uparrow}^{>} G_{L \downarrow}^{<}\left(1+\Sigma_{R}^{a} G_{d}^{a}\right) + G_{d}^{r}\Sigma_{R}^{<} G_{d}^{a}
\end{split}
\end{equation}
Similarly,
\begin{equation}
G_{d}^{>}=-i\left(G_{d}^{r} \Sigma_{R}^{r}+1\right) G_{L \uparrow}^{<} G_{L \downarrow}^{>}\left(1+\Sigma_{R}^{a} G_{d}^{a}\right)+G_{d}^{r} \Sigma_{R}^{>} G_{d}^{a}
\end{equation}
\subsection{DC spin current}

\begin{equation}
I_{s}= 2\rm{Re}\sum_{q} \int  \frac{dE}{2\pi}\operatorname{Tr}\left[\left(G_{d}^{>}-G_{d}^{<}\right) \Sigma_{R q}^{<}+G_{d}^{<}\left(\Sigma_{R q}^{a}-\Sigma_{R q}^{r}\right)\right]
\end{equation}

We have
\begin{equation}
G_{d}^{>}(E)-G_{d}^{<}(E)=-i\left(G_{d}^{r} \Sigma_{R}^{r}+1\right)\left(G_{L \uparrow}^{<} G_{L \downarrow}^{>}-G_{L \uparrow}^{>} G_{L \downarrow}^{<}\right)\left(1+\Sigma_{R}^{a} G_{d}^{a}\right)+G_{d}^{r}\left(\Sigma_{R}^{>}-\Sigma_{R}^{<}\right) G_{d}^{a}
\end{equation}
Fourier transformation
\begin{equation}
G_{d}^{<}(E) = \int_{-\infty}^{+\infty} dt G_{d}^{<}(t-t')e^{iE(t-t')}
\end{equation}
and inverse Fourier transformation
\begin{equation}
G_{d}^{<}(t-t') = \frac{1}{2\pi}\int_{-\infty}^{+\infty} d\omega G_{d}^{<}(E)e^{-iE(t-t')},
\end{equation}
are used, since the Green's functions only dependent on time difference. Then using Keldysh equation, we have
\begin{equation}
  \label{eq:3}
G_{L,mn\sigma}^{<}(E) = G_{L,mn\sigma}^{r}\Sigma_{L,mn\sigma}^{<}(E) G_{L,mn\sigma}^{a}(E),
\end{equation}
where $G_{L,mn\sigma}$ is the Green's function when left free lead, QD and left coupling present.  $\Sigma_{L,mn\sigma}^{<}$ is self-energy of left lead, defined in Eq. (\ref{eq:left-self-energy})
\begin{equation}
\Sigma_{L,mn\sigma}^{<} = if_{L\sigma}(E) \Gamma_{L,mn\sigma}(E).
\end{equation}
so,
\begin{equation}
G_{L,mn\sigma}^{<}(E) = iG_{L,mn\sigma}^{r}f_{L\sigma}(E) \Gamma_{L,mn\sigma}(E) G_{L,mn\sigma}^{a}(E)  \equiv i D_{L \sigma} f_{L \sigma},
\end{equation}
and
\begin{equation}
\begin{split}
G_{L,mn\sigma}^{>}(E) &= -(G_{L,mn\sigma}^{<}(E))^{\dag} \\
&= G_{L,mn\sigma}^{r}(E)\Sigma_{L,mn\sigma}^{>}(E) G_{L,mn\sigma}^{a}(E)\\
&=iD_{L\sigma}(f_{L\uparrow}(E) - 1)
\end{split}
\end{equation}
in which, $D_{L\sigma} = G_{L\sigma}^{r}\Gamma_{L\sigma}G_{L\sigma}^{a}$, thus
\begin{equation}
\begin{split}
G_{L\sigma}^{<} G_{L\sigma}^{>} - G_{L\sigma}^{>} G_{L\sigma}^{<} &= D_{L\uparrow}D_{L\downarrow} [(f_{L\uparrow}-1)f_{L\downarrow} - (f_{L\downarrow}-1) f_{L\uparrow}] \\
&= D_{L\uparrow}D_{L\downarrow} (f_{L\uparrow} - f_{L\downarrow})
\end{split}
\end{equation}
\begin{equation}
\begin{split}
\Sigma_{R}^{<}(E) &=  \sum_{q_{1}}J_{q_{1}}^{2} g_{Rq_{1}}^{<}(E) \\
&= if_{R}^{B}(E) \Gamma_{R}(E)
\end{split}
\end{equation}
\begin{equation}
\Sigma_{R}^{a} - \Sigma_{R}^{r} = \Sigma_{R}^{<} - \Sigma_{R}^{>} = i\Gamma_{R}(E).
\end{equation}
\begin{equation}
G_{d}^{>}-G_{d}^{<}=-i\left[f_{L \uparrow}-f_{L \downarrow}\right]\left(G_{d}^{r} \Sigma_{R q}^{r}+1\right) D_{L \uparrow} D_{L \downarrow}\left(1+\Sigma_{R q}^{a} G_{d}^{a}\right)-i G_{d}^{r} \Gamma_{R q} G_{d}^{a}
\end{equation}
\begin{equation}
\begin{aligned}
\left(G_{d}^{>}-G_{d}^{<}\right) \Sigma_{R q}^{<}+G_{d}^{<}\left(\Sigma_{R q}^{a}-\Sigma_{R q}^{r}\right) &=\left[\left(f_{L \uparrow}-f_{L \downarrow}\right) f_{R}+\left(f_{L \uparrow}-1\right) f_{L \downarrow}\right] \\ & \times\left(G_{d}^{r} \Sigma_{R q}^{r}+1\right) D_{L \uparrow} D_{L \downarrow}\left(1+\Sigma_{R q}^{a} G_{d}^{a}\right) \Gamma_{R q} 
\end{aligned}
\label{eq:diff}
\end{equation}
The following formula exists
\begin{equation}
[f_{L \uparrow}(\varepsilon)-1] f_{L \downarrow}(\varepsilon) =-[f_{L \uparrow}(\varepsilon) -f_{L \downarrow}(\varepsilon)] f_{L}^{B}
\end{equation}
where,
\begin{equation}
f_{L \sigma}(\epsilon)=\frac{1}{e^{\beta_{L}(\epsilon-\mu_{\sigma})}+1}
\end{equation}
\begin{equation}
f_{L}^{B}= \frac{1}{e^{\beta_{L}\Delta \mu_{s}}-1 }
\end{equation}
$\Delta\mu_{s}=\mu_{\uparrow}-\mu_{\downarrow}$ ($\omega = \varepsilon_{\downarrow} - \varepsilon_{\uparrow}$?). Note that this similar relation also exists,
\begin{equation}
\left(f_{L \uparrow}(\varepsilon)-1\right) f_{L \downarrow}(\varepsilon+\omega)=-[f_{L \uparrow}(\varepsilon) - f_{L \downarrow}(\varepsilon+\omega)] f_{L}^{B}(\omega)
\end{equation}
\begin{equation}
f_{L}^{B}(\varepsilon) = \frac{1}{e^{\beta_{L}(\omega+\Delta \mu_{s})}-1 },
\end{equation}
is the effective Boson-Einstein distribution of left electronic lead. Eq. (\ref{eq:diff}) becomes
\begin{equation}
\begin{aligned}
\left(G_{d}^{>}-G_{d}^{<}\right) \Sigma_{R q}^{<}+G_{d}^{<}\left(\Sigma_{R q}^{a}-\Sigma_{R q}^{r}\right) &=\left[\left(f_{L \uparrow}-f_{L \downarrow}\right) (f_{R} -  f_{L}^{B}) \right] \\ & \times\left(G_{d}^{r} \Sigma_{R q}^{r}+1\right) D_{L \uparrow} D_{L \downarrow}\left(1+\Sigma_{R q}^{a} G_{d}^{a}\right) \Gamma_{R q} 
\end{aligned}
\end{equation}
Substitute in Eq. (\ref{eq:current1}), we get
\begin{equation}
I_{s R}=\int d \omega \rho_{R}(\omega)\left(f_{R}(\omega)-f_{L}^{B}(\omega)\right) \int d E\left(f_{L \uparrow}(E)-f_{L \downarrow}(E+\omega)\right) \operatorname{Tr}[A(E, \omega)],
\end{equation}
\begin{equation}
A(E, \omega)=\left[G_{d}^{r}(E) \Sigma_{R q}^{r}(\omega)+1\right] D_{L \uparrow}(E) D_{L \downarrow}(E+\omega)\left[1+\Sigma_{R q}^{a}(\omega) G_{d}^{a}(E)\right].
\end{equation}
Above $\rho_{R}(\omega)$ comes from the magnon $q$ summation, is density of states of magnon lead, determined by magnon dispersion $\omega_{q}$.
\subsection{Spin current from the left lead}
Define spin density operator
\begin{equation}
N_{sk} = d_{k \uparrow}^{\dagger} d_{k \uparrow}-d_{k \downarrow}^{\dagger} d_{k \downarrow}
\end{equation}
\begin{equation}
I_{s L}=(1 / 2) \partial_{t} N_{s}=(1 / 2)\left(I_{\uparrow}-I_{\downarrow}\right)
\end{equation}
\begin{equation}
I_{\sigma}=\operatorname{Tr}\left[\left(G_{d \sigma}^{r}-G_{d \sigma}^{a}\right) \Sigma_{L \sigma}^{<}+G_{d \sigma}^{<}\left(\Sigma_{L \sigma}^{a}-\Sigma_{L \sigma}^{r}\right)\right]
\end{equation}
\begin{equation}
\left[G_{d \sigma}\right]_{n m}=-i\left\langle T_{c} S d_{n \sigma} d_{m \sigma}^{\dagger}\right\rangle
\end{equation}
the factor of 1/2 comes from spin of electron while spin of magnon is 1.






\begin{thebibliography}{10}
\bibitem{ref1}
Y, K, Kato. Observation of the Spin Hall Effect in Semiconductors[J]. Science, 2004.
\bibitem{CaoZhan}
Cao Zhan, Investigation on DC electronic transport in hybrid multiterminal quantum dot systems[D], 2017.
\end{thebibliography}

\end{CJK}
\end{document}

\documentclass[11pt,a4paper]{article}
\usepackage{CJKutf8}
\usepackage{graphicx}
\usepackage{amsmath}
\usepackage{braket}
\usepackage{caption}
\usepackage{bm}
\captionsetup{font={scriptsize}}
\begin{CJK}{UTF8}{gbsn}
\begin{document}

\title{formula in PRB. 88, 220406(R) (2013)}
\author{Li Gaoyang}
%\date{\today}
\date{\today}
\maketitle
\tableofcontents

% \newpage
\section{formulas in PRB. 88, 220406(R) (2013)}
\subsection{formula 1}
System Hamiltonion:
\begin{equation}
H=H_{L}+H_{R}+H_{s d}.
\end{equation}
Left lead is metallic
\begin{equation}
H_{L} = \sum_{k\sigma} \left(\varepsilon_{k\sigma}-\mu_{\sigma}\right) c_{k \sigma}^{\dagger} c_{k \sigma},
\end{equation}
right lead is insulating magnetic
\begin{equation}
H_{R} \approx \sum_{q} \hbar w_{q} a_{q}^{\dagger} a_{q}+\text { constant }.
\end{equation}
The interfacial electron-magnon interaction is described by
\begin{equation}
H_{s d}=-\sum_{k, q} J_{q}\left[S_{q}^{-} c_{k \uparrow}^{\dagger} c_{k+q \downarrow}+S_{q}^{+} c_{k+q \downarrow}^{\dagger} c_{k \uparrow}\right]
\end{equation}
where $S_{q}^{-} \approx \sqrt{2 S_{0}} a_{q}^{\dagger}, S_{q}^{+} \approx \sqrt{2 S_{0}} a_{q}$ are in the momentum space and $J_{q}$ denotes the effective exchange coupling at the interface. The magnonic spin current operator can be obtained by
\begin{equation}
\hat{I}_{S} = \frac{d\hat{N}_{R}}{dt} =  \frac{d}{dt} \sum_{q}a_{q}^{\dag}a_{q},
\end{equation}
the magnonic spin current is obtained by taking average over the nonequilibrium ground state $\ket{\psi_{0}}$ of the interacting system $H$:
\begin{equation}
I_{S} = \frac{dN_{R}}{dt} =  \frac{d}{dt} \langle \sum_{q}a_{q}^{\dag}a_{q}\rangle.
\end{equation}
Using the Heisenberg equation, we get
\begin{equation}
I_{S}=\frac{i}{\hbar}\langle[H_{s d}, \sum_{q} a_{q}^{\dagger} a_{q}]\rangle.
\end{equation}
\begin{equation}
[H_{s d}, \sum_{q} a_{q}^{\dagger} a_{q}] = [-\sum_{k, q} J_{q}\left(S_{q}^{-} c_{k \uparrow}^{\dagger} c_{k+q \downarrow}+S_{q}^{+} c_{k+q \downarrow}^{\dagger} c_{k \uparrow}\right), \sum_{q} a_{q}^{\dagger} a_{q}],
\end{equation}
in which,
\begin{equation}
[a_{q}^{\dag}, \sum_{q^{\prime}} a_{q'}^{\dagger}, a_{q^{\prime}}] = \delta_{q q^{\prime}}[a_{q}^{\dagger}, a_{q^{\prime}}^{\dagger} a_{q^{\prime}}] = [a_{q}^{\dagger}, a_{q}^{\dagger} a_{q}] = a_{q}^{\dagger}[a_{q}^{\dagger}, a_{q}] = - a_{q}^{\dagger}.
\end{equation}
Similarly,
\begin{equation}
[a_{q}, \sum_{q^{\prime}} a_{q^{\prime}}^{+} a_{q^{\prime}}]=[a_{q}, a_{q}^{\dagger} a_{q}]=a_{q}.
\label{eq:1-1}
\end{equation}
So,
\begin{equation}
\begin{split}
I_{S}&=\frac{i}{\hbar}\langle -\sum_{k q} J_{q}\left(-S_{q}^{-} c_{k \uparrow}^{\dagger} c_{k+q\downarrow} + S_{q}^{+} c_{k+q\downarrow}^{\dagger} c_{k \uparrow}\right) \rangle \\
&= \frac{i}{\hbar} \sum_{k q} J_{q}\left( \langle S_{q}^{-} c_{k \uparrow}^{\dagger} c_{k+q\downarrow}\rangle - \langle S_{q}^{+} c_{k+q\downarrow}^{\dagger} c_{k \uparrow}\rangle \right).
\end{split}
\end{equation}
\subsection{formula 2}
\begin{equation}
\frac{d}{d t}\left\langle S_{q}^{+} c_{k+q \downarrow}^{\dagger} c_{k \uparrow}\right\rangle=\frac{i}{\hbar}\left\langle\left[H_{L}+H_{s d}+H_{R}, S_{q}^{+} c_{k+q \downarrow}^{\dagger} c_{k \uparrow}\right]\right\rangle.
\label{eq:2-1}
\end{equation}
The rhs. of eq. (\ref{eq:2-1}) is decomposed into 3 terms. The first term reads
\begin{equation}
\begin{split}
\left\langle\left[H_{L}, S_{q}^{+} c_{k+q \downarrow}^{\dagger} c_{k \uparrow}\right]\right\rangle &= [\sum_{k'\sigma}\left(\varepsilon_{k^{\prime}\sigma}-\mu_{\sigma}\right) c_{k' \sigma}^{\dagger} c_{k' \sigma}, S_{q}^{+} c_{k+q \downarrow}^{\dagger} c_{k \uparrow}] \\
&= S_{q}^{+} [\sum_{k'\sigma}\left(\varepsilon_{k^{\prime}\sigma}-\mu_{\sigma}\right) c_{k' \sigma}^{\dagger} c_{k' \sigma}, c_{k+q \downarrow}^{\dagger}]c_{k \uparrow} \\
&\quad + S_{q}^{+}c_{k+q \downarrow}^{\dagger}[\sum_{k'\sigma}\left(\varepsilon_{k^{\prime}\sigma}-\mu_{\sigma}\right) c_{k' \sigma}^{\dagger} c_{k' \sigma}, c_{k \uparrow}]
\end{split}
\end{equation}
Note that,
\begin{equation}
[\sum_{k^{\prime} \sigma} c_{k^{\prime} \sigma}^{\dag} c_{k^{\prime} \sigma}, c_{k+q \downarrow}^{\dagger}] = \sum_{k'\sigma} c_{k'\sigma}^{\dag}\delta_{k',k+q}\delta_{\sigma\downarrow} = c_{k+q\downarrow}^{\dag},
\label{eq:2-2}
\end{equation}
\begin{equation}
[\sum_{k^{\prime} \sigma} c_{k^{\prime} \sigma}^{\dag} c_{k^{\prime} \sigma}, c_{k \uparrow}] = - \sum_{k^{\prime} \sigma} \{c_{k'\sigma}^{\dag}, c_{k\uparrow}\}c_{k'\sigma} = - c_{k\uparrow}.
\label{eq:2-3}
\end{equation}
Eq. (\ref{eq:2-2}) (\ref{eq:2-3}) are derived using equity
\begin{equation}
\begin{split}
[AB, C] &= A[B, C] + [A, C]B \\
&= A\{B, C\} - \{A, C\}B.
\end{split}
\end{equation}
So,
\begin{equation}
\begin{split}
\left[H_{L}, S_{q}^{+} c_{k+q \downarrow}^{\dagger} c_{k \uparrow}\right] &= \left(\varepsilon_{k+q \downarrow}-\varepsilon_{k \uparrow}+\mu_{\uparrow}-\mu_{\downarrow}\right)S_{q}^{+} c_{k+q \downarrow}^{\dagger} c_{k \uparrow}.
\end{split}
\label{eq:1-0}
\end{equation}
If $\mu_{\uparrow} = \mu_{\downarrow}$, then eq. (\ref{eq:1-0}) reduces to
\begin{equation}
\begin{split}
\left[H_{L}, S_{q}^{+} c_{k+q \downarrow}^{\dagger} c_{k \uparrow}\right] &= \left(\varepsilon_{k+q \downarrow}-\varepsilon_{k \uparrow}\right)S_{q}^{+} c_{k+q \downarrow}^{\dagger} c_{k \uparrow}.
\end{split}
\label{eq:1-01}
\end{equation}
The second term $H_{R} = \sum_{q} \hbar w_{q} a_{q}^{\dagger} a_{q}$, then using eq. (\ref{eq:1-1}), we get
\begin{equation}
\left[H_{R}, S_{q}^{+} c_{k+q \downarrow}^{\dagger} c_{k \uparrow}\right] = -\hbar \omega_{q}S_{q}^{+} c_{k+q \downarrow}^{\dagger} c_{k \uparrow}.
\end{equation}
The third term in eq. (\ref{eq:2-1}) reads
\begin{equation}
\begin{split}
\left[H_{sd}, S_{q}^{+} c_{k+q \downarrow}^{\dagger} c_{k \uparrow}\right] &= - J_{q} \left[S_{q}^{-} c_{k \uparrow}^{\dagger} c_{k+q \downarrow}, S_{q}^{+} c_{k+q \downarrow}^{\dagger} c_{k \uparrow}\right] \\
&= J_{q} \left[S_{q}^{+} c_{k+q \downarrow}^{\dagger} c_{k \uparrow}, S_{q}^{-} c_{k \uparrow}^{\dagger} c_{k+q \downarrow}\right]
\end{split}
\end{equation}
Combine these three terms, we get
\begin{equation}
\begin{split}
\frac{d}{d t}\left\langle S_{q}^{+} c_{k+q \downarrow}^{\dagger} c_{k \uparrow}\right\rangle&=\frac{i}{\hbar}\left(\varepsilon_{k+q \downarrow}-\varepsilon_{k \uparrow}-\hbar \omega_{q}\right)\left\langle S_{q}^{+} c_{k+q \downarrow}^{\dagger} c_{k \uparrow}\right\rangle\\ 
&\quad + \frac{i}{\hbar} J_{q}\left\langle\left[S_{q}^{+} c_{k+q \downarrow}^{\dagger} c_{k \uparrow}, S_{q}^{-} c_{k \uparrow}^{\dagger} c_{k+q \downarrow}\right]\right\rangle,
\end{split}
\end{equation}
which is also eq. (2) in PRB. 88, 220406(R) (2013).

\section{NM-QD-MIL system}
For system consists of quantum dot(QD) sandwiched by a left normal metal lead and a right magnetic insulating lead(MIL), the Hamiltonian is
\begin{equation}
H=H_{\rm{L}}+H_{\rm{QD}}+H_{\rm{R}}+H_{\rm{T}}.
\end{equation}

\begin{equation}
H_{\rm{L}} = \sum_{k\sigma} \varepsilon_{k\sigma} c_{k \sigma}^{\dagger} c_{k \sigma},
\end{equation}

\begin{equation}
H_{\rm{QD}} = \sum_{\sigma}\varepsilon_{\sigma} d_{\sigma}^{\dagger} d_{\sigma},
\end{equation}

\begin{equation}
H_{\rm{R}} \approx \sum_{q} \hbar w_{q} a_{q}^{\dagger} a_{q},
\end{equation}

\begin{equation}
H_{\rm{T}} = V_{\rm{L}} + V_{\rm{R}}
\end{equation}
V$_{L}$ is the coupling between left lead and QD, while V$_{R}$ is the coupling between right lead and QD.
\begin{equation}
V_{\rm{L}} = \sum_{k\sigma}( t_{k\sigma} c_{k \sigma}^{\dagger} d_{\sigma} + t_{k\sigma}^{*} d_{\sigma}^{\dagger} c_{k\sigma})
\end{equation}

\begin{equation}
V_{\rm{R}}=-\sum_{q} J_{q}\left[S_{q}^{-} d_{\uparrow}^{\dagger} d_{\downarrow}+S_{q}^{+} d_{\downarrow}^{\dagger} d_{\uparrow}\right].
\end{equation}

where $S_{q}^{-} \approx \sqrt{2 S_{0}} a_{q}^{\dagger}, S_{q}^{+} \approx \sqrt{2 S_{0}} a_{q}$ are in the momentum space and $J_{q}$ denotes the effective exchange coupling between the QD and MIL.
\subsection{Spin-dependent current in left lead}
The spin-dependent current flow out of left lead is $I_{L\sigma}$:
\begin{equation}
I_{L\sigma} = \frac{d}{dt} \langle N_{L\sigma} \rangle
\end{equation}
in which, $N_{L\sigma} = \sum_{k} c_{k\sigma}^{\dag}c_{k\sigma}$
Heisenberg equation:
\begin{equation}
\frac{d}{dt} \langle N_{L\sigma} \rangle = \frac{i}{\hbar} \langle [H, N_{L\sigma}]\rangle
\end{equation}

\begin{equation}
[H, N_{L\sigma}] = [H_{T}, N_{L\sigma}] =  \sum_{k}\left(t_{k \sigma}^{*} d_{\sigma}^{\dagger} c_{k \sigma} - t_{k \sigma} c_{k \sigma}^{\dagger} d_{\sigma} \right)
\end{equation}
so, the spin-dependent current
\begin{equation}
I_{L\sigma} = \frac{i}{\hbar} \sum_{k} \left( t_{k \sigma}^{*} \langle  d_{\sigma}^{\dagger} c_{k \sigma}\rangle - t_{k \sigma} \langle c_{k \sigma}^{\dagger} d_{\sigma} \rangle \right).
\end{equation}
Namely, the spin-up current is
\begin{equation}
I_{L\uparrow} = \frac{i}{\hbar} \sum_{k} \left( t_{k \uparrow}^{*} \langle  d_{\uparrow}^{\dagger} c_{k \uparrow}\rangle - t_{k \uparrow} \langle c_{k \uparrow}^{\dagger} d_{\uparrow} \rangle \right),
\end{equation}
the spin-down current is
\begin{equation}
I_{L\downarrow} = \frac{i}{\hbar} \sum_{k} \left( t_{k \downarrow}^{*} \langle  d_{\downarrow}^{\dagger} c_{k \downarrow}\rangle - t_{k \downarrow} \langle c_{k \downarrow}^{\dagger} d_{\downarrow} \rangle \right),
\end{equation}
The charge current in left lead is defined as
\begin{equation}
I_{e} = e(I_{L\uparrow} + I_{L\downarrow}).
\end{equation}
The spin current in left lead is defined as
\begin{equation}
I_{LS} = \frac{1}{2}(I_{L\uparrow} - I_{L\downarrow})
\end{equation}
Substitute the spin-dependent current in, we get the spin current in left lead
\begin{equation}
I_{LS} = \frac{i}{2\hbar} \sum_{k} \left( t_{k \uparrow}^{*} \langle  d_{\uparrow}^{\dagger} c_{k \uparrow}\rangle - t_{k \uparrow} \langle c_{k \uparrow}^{\dagger} d_{\uparrow} \rangle -  t_{k \downarrow}^{*} \langle  d_{\downarrow}^{\dagger} c_{k \downarrow}\rangle + t_{k \downarrow} \langle c_{k \downarrow}^{\dagger} d_{\downarrow} \rangle \right)
\end{equation}
\subsection{magnonic current in right lead}
The magnonic current in right lead is
\begin{equation}
I_{RS} = \frac{d\langle N_{R}\rangle}{dt} =  \frac{d}{dt} \langle \sum_{q}a_{q}^{\dag}a_{q}\rangle,
\end{equation}
From the Heisenberg equation, we have
\begin{equation}
\frac{d}{dt} \langle \sum_{q}a_{q}^{\dag}a_{q}\rangle = \frac{i}{\hbar} \langle [H, \sum_{q}a_{q}^{\dag}a_{q}] \rangle.
\end{equation}
We have
\begin{equation}
\begin{split}
&[H, \sum_{q}a_{q}^{\dag}a_{q}] \\
= &[V_{R}, \sum_{q}a_{q}^{\dag}a_{q}]\\
= &-\sum_{q} J_{q} \Big(\big[S_{q}^{-}d_{\uparrow}^{\dag}d_{\downarrow}, \sum_{q'}a_{q'}^{\dag}a_{q'}\big] + \big[S_{q}^{+}d_{\downarrow}^{\dag}d_{\uparrow}, \sum_{q'}a_{q'}^{\dag}a_{q'}\big]\Big) \\
= & \sum_{q} J_{q} \big( S_{q}^{-}d_{\uparrow}^{\dag}d_{\downarrow} - S_{q}^{+}d_{\downarrow}^{\dag}d_{\uparrow} \big)
\end{split}
\end{equation}
So, the magnon current reads
\begin{equation}
I_{RS} = \frac{i}{\hbar} \sum_{q} J_{q} \big( \big\langle S_{q}^{-}d_{\uparrow}^{\dag}d_{\downarrow}\big\rangle - \big\langle S_{q}^{+}d_{\downarrow}^{\dag}d_{\uparrow} \big\rangle \big) .
\end{equation}

\subsection{spin-dependent current in QD}
Similarly, the spin-dependent current in the central QD is defined as
\begin{equation}
I_{C\sigma} = \langle\frac{dN_{C\sigma}}{dt} \rangle= \langle  \frac{d}{dt} d_{\sigma}^{\dag}d_{\sigma}\rangle,
\end{equation}
Heisenberg equation:
\begin{equation}
\frac{d}{dt} d_{\sigma}^{\dag}d_{\sigma} = \frac{i}{\hbar} [H, d_{\sigma}^{\dag}d_{\sigma}].
\end{equation}
Specifically, we have
\begin{equation}
 [H, d_{\sigma}^{\dag}d_{\sigma}] = [V_{L},  d_{\sigma}^{\dag}d_{\sigma}] + [V_{R},  d_{\sigma}^{\dag}d_{\sigma}]
\end{equation}
in which,
\begin{equation}
\begin{split}
[V_{L},  d_{\sigma}^{\dag}d_{\sigma}] &= \sum_{k'\sigma'}\big(t_{k'\sigma'} \big[c_{k'\sigma'}^{\dag}d_{\sigma'},  d_{\sigma}^{\dag}d_{\sigma}\big] + t_{k'\sigma'}^{*} \big[d_{\sigma'}^{\dag}c_{k'\sigma'},  d_{\sigma}^{\dag}d_{\sigma}\big] \big) \\
&  = \sum_{k'\sigma'}\big(t_{k'\sigma'} c_{k'\sigma'}^{\dag}d_{\sigma'}\delta_{\sigma\sigma'} - t_{k'\sigma'}^{*} d_{\sigma'}^{\dag}c_{k'\sigma'} \delta_{\sigma\sigma'}\big) \\
&= \sum_{k}\big(t_{k\sigma} c_{k\sigma}^{\dag}d_{\sigma} - t_{k\sigma}^{*} d_{\sigma}^{\dag}c_{k\sigma} \big)
\end{split}
\end{equation}
We change the summation index from $k'$ to $k$ in the last line , which doesn't change the result.
\begin{equation}
[V_{R},  d_{\sigma}^{\dag}d_{\sigma}] = -\sum_{q}J_{q}\big( S_{q}^{-} \big[d_{\uparrow}^{\dag}d_{\downarrow}, d_{\sigma}^{\dag}d_{\sigma} \big] + S_{q}^{+}\big[d_{\downarrow}^{\dag}d_{\uparrow}, d_{\sigma}^{\dag}d_{\sigma} \big]\big)
\end{equation}
in which,
\begin{equation}
\begin{split}
\big[d_{\uparrow}^{\dag}d_{\downarrow}, d_{\sigma}^{\dag}d_{\sigma} \big] &= d_{\uparrow}^{\dag} \big[d_{\downarrow}, d_{\sigma}^{\dag}d_{\sigma}\big] + \big[d_{\uparrow}^{\dag} , d_{\sigma}^{\dag}d_{\sigma} \big]d_{\downarrow} \\
&=d_{\uparrow}^{\dag}d_{\downarrow}\delta_{\sigma \downarrow} - d_{\uparrow}^{\dag}d_{\downarrow}\delta_{\sigma \uparrow}
\end{split}
\end{equation}
So,
\begin{equation}
[V_{R},  d_{\sigma}^{\dag}d_{\sigma}] =  -\sum_{q}J_{q}\big[ S_{q}^{-} \big( d_{\uparrow}^{\dag}d_{\downarrow}\delta_{\sigma \downarrow} - d_{\uparrow}^{\dag}d_{\downarrow}\delta_{\sigma \uparrow} \big) + S_{q}^{+}\big(d_{\downarrow}^{\dag}d_{\uparrow}\delta_{\sigma \uparrow} - d_{\downarrow}^{\dag}d_{\uparrow}\delta_{\sigma \downarrow} \big)\big]
\end{equation}
The spin-dependent current is
\begin{eqnarray}
I_{C\uparrow} = \frac{i}{\hbar}\big[ H, d_{\uparrow}^{\dag} d_{\uparrow} \big]
\\
I_{C\downarrow} = \frac{i}{\hbar}\big[ H, d_{\downarrow}^{\dag} d_{\downarrow} \big]
\end{eqnarray}
which gives
\begin{equation}
I_{C\uparrow} = \frac{i}{\hbar} \big[ \sum_{k}\big(t_{k\uparrow} c_{k\uparrow}^{\dag}d_{\uparrow} - t_{k\uparrow}^{*} d_{\uparrow}^{\dag}c_{k\uparrow} \big) + \sum_{q}J_{q}\big( S_{q}^{-} d_{\uparrow}^{\dag}d_{\downarrow} - S_{q}^{+}d_{\downarrow}^{\dag}d_{\uparrow} \big)\big]
\end{equation}
and
\begin{equation}
I_{C\downarrow} = \frac{i}{\hbar} \big[ \sum_{k}\big(t_{k\downarrow} c_{k\downarrow}^{\dag}d_{\downarrow} - t_{k\downarrow}^{*} d_{\downarrow}^{\dag}c_{k\downarrow} \big) + \sum_{q}J_{q}\big(S_{q}^{+}d_{\downarrow}^{\dag}d_{\uparrow} -  S_{q}^{-} d_{\uparrow}^{\dag}d_{\downarrow} \big)\big].
\end{equation}
The spin current in central dot is
\begin{equation}
I_{CS} = \frac{1}{2}(I_{C\uparrow} - I_{C\downarrow})
\end{equation}

\subsection{Verifying continuity condition}
\subsubsection{Charge current}
Since the right lead is a insulating lead, there is no charge current flow through it, so the charge current is
\begin{equation}
I_{Re} = 0
\end{equation}
where the subscript R denotes the right lead, while the subscript e denotes the charge current.

Meanwhile, the charge current flows in left lead and QD is
\begin{equation}
I_{e} = e(\sum_{\sigma}I_{L\sigma} + \sum_{\sigma}I_{C\sigma}) = 0
\end{equation}
\subsubsection{Spin current}
Spin current in the left lead and QD:
\begin{equation}
I_{LS} + I_{CS} = \frac{i}{\hbar} \sum_{q} J_{q} \big( \big\langle S_{q}^{-}d_{\uparrow}^{\dag}d_{\downarrow}\big\rangle - \big\langle S_{q}^{+}d_{\downarrow}^{\dag}d_{\uparrow} \big\rangle \big) .
\end{equation}
The magnon current is
\begin{equation}
I_{RS} = \frac{i}{\hbar} \sum_{q} J_{q} \big( \big\langle S_{q}^{-}d_{\uparrow}^{\dag}d_{\downarrow}\big\rangle - \big\langle S_{q}^{+}d_{\downarrow}^{\dag}d_{\uparrow} \big\rangle \big) .
\end{equation}
Thus, we have
\begin{equation}
\begin{split}
I_{LS} + I_{CS} &= I_{RS}\\
& = \frac{i}{\hbar} \sum_{q} J_{q} \big( \big\langle S_{q}^{-}d_{\uparrow}^{\dag}d_{\downarrow}\big\rangle - \big\langle S_{q}^{+}d_{\downarrow}^{\dag}d_{\uparrow} \big\rangle \big) .
\end{split}
\end{equation}
\section{Nonequilibrium Green's function technique}


\end{CJK}
\end{document}

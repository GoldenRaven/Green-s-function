\documentclass[aps,prb,superscriptaddress]{revtex4-2}
\usepackage{xcolor}
\usepackage{CJKutf8}
\usepackage{graphicx}
\usepackage{amsmath}
\usepackage{color}
\usepackage{braket}
\usepackage{caption}
\usepackage{bm}
\usepackage[colorlinks=true, letterpaper=true, pdfstartview=FitV, linkcolor=blue, citecolor=blue, urlcolor=blue]{hyperref}
% \usepackage{geometry}
% \geometry{a4paper,scale=0.8}

\captionsetup{font={scriptsize}}

% \begin{CJK}{UTF8}{gbsn}
\begin{document}

\title{Notes on quantum transport in mesoscopic systems}
\author{Li Gaoyang}
\date{2020/12/1}
% \date{\today}
\maketitle
% \tableofcontents

\section{Nonequilibrium Green's function technique}
\subsection{Demonstrative Hamiltonian}
\begin{equation}
\hat{H}=H_{l e a d}+H_{d o t}+H_{T}
\end{equation}
\begin{equation}
H_{l e a d}=\sum_{k \alpha} \epsilon_{k \alpha} \hat{C}_{k \alpha}^{\dagger} \hat{C}_{k \alpha}
\end{equation}
\begin{equation}
\epsilon_{k \alpha}=\epsilon_{k \alpha}^{(0)}+q v_{\alpha}
\end{equation}
\begin{equation}
H_{d o t}=\sum_{n}\left(\epsilon_{n}+q U_{n}\right) d_{n}^{\dagger} d_{n}
\end{equation}
\begin{equation}
U_{n}=\sum_{m} V_{n m}<d_{m}^{\dagger} d_{m}>
\end{equation}
\begin{equation}
H_{T}=\sum_{k \alpha n}\left[t_{k \alpha n} \hat{C}_{k \alpha}^{\dagger} \hat{d}_{n}+t_{k \alpha n}^{*} \hat{d}_{n}^{\dagger} \hat{C}_{k \alpha}\right]
\end{equation}
\subsection{Current definition}
We use the Hamiltonian in WangJian's notes. Equation of motion of particle operator $\hat{N}_{\alpha k\sigma}$ in the lead $\alpha$ is
\begin{equation}
\begin{split}
\frac{d}{dt}\hat{N}_{\alpha} &= \frac{i}{\hbar}[H, \sum_{k}c_{\alpha k}^{\dag}c_{\alpha k}] = \left[\sum_{k'n, \alpha'=L, R}\left[t_{k' \alpha'} c_{k' \alpha'}^{\dag} d_{n}+\mathrm{c.c.}\right], \sum_{k}c_{\alpha k}^{\dag}c_{\alpha k}\right]\\
&=\frac{i}{\hbar}\sum_{kk',n, \alpha'=L, R}\left[ -t_{k' \alpha'} c_{k' \alpha' }^{\dag} d_{n}\delta_{\alpha\alpha'}\delta_{kk'}+\mathrm{c.c.}\right]\\
&=\frac{i}{\hbar}\sum_{kn}[-t_{k \alpha} c_{k \alpha}^{\dag} d_{n} + t_{k \alpha}^{*} d_{n}^{\dag}c_{k \alpha}]
\end{split}
\end{equation}
So, the charge current is given by
\begin{equation}
\begin{split}
I_{\alpha}(t) &= e\langle\frac{d}{dt}\hat{N}_{\alpha}(t)\rangle \\
&=\frac{ie}{\hbar}\sum_{kn}(\langle -t_{k \alpha} c_{k \alpha}^{\dag}(t) d_{n}(t)\rangle + \langle t_{k \alpha}^{*} d_{n}^{\dag}(t)c_{k \alpha}(t)\rangle)
\end{split}
\end{equation}
Define the lesser Green's function
\begin{equation}
G_{\sigma',k\alpha\sigma}^{<}(t,t') = i\langle c_{k\alpha\sigma}^{\dag}(t') d_{\sigma'}(t)\rangle 
\end{equation}
the charge current is written as
\begin{equation}
I_{L}(t)=\frac{-e}{\hbar}\sum_{kn\alpha\in L}(t_{k\alpha n}G_{n,k\alpha\sigma}^{<}(t,t) - t_{k \alpha n}^{*} G_{k\alpha,n}(t,t)\rangle)
\end{equation}
More generally, we define the contour Green's function
\begin{equation}
G_{n,k\alpha}(\tau,\tau') = -i\langle  d_{n}(\tau) c_{k\alpha}^{\dag}(\tau')\rangle .
\end{equation}
Following Jauho's notation~\cite{Jauho}, when the electron in the lead is non-interacting, G$_{n,k\alpha\sigma}(\tau,\tau')$ is related to $G_{nm}$ and $g_{k\alpha}$ by the following contour integral
\begin{equation}
G_{n,k\alpha}(\tau,\tau')=\sum_{m} \int d \tau_{1} G_{n m}\left(\tau, \tau_{1}\right) t_{k \alpha m}^{*} g_{k \alpha}\left(\tau_{1}, \tau^{\prime}\right)
\end{equation}
where
\begin{equation}
G_{n m}\left(\tau_{1}, \tau_{2}\right) \equiv-i \langle T_{c} \left[d_{n} \left(\tau_{1}\right) d_{m}^{\dagger}\left(\tau_{2}\right)\right]\rangle
\end{equation}
\begin{equation}
g_{k \alpha}\left(\tau_{1}, \tau_{2}\right) \equiv-i \langle T_{c}\left[c_{k \alpha}\left(\tau_{1}\right) c_{k \alpha}^{\dagger}\left(\tau_{2}\right)\right] \rangle_{0}.
\end{equation}
Using the theorem of analytic continuation, we have
\begin{equation}
\begin{aligned}
G_{n, k \alpha}^{<}\left(t, t^{\prime}\right) &=\sum_{m} \int d t_{1}\left[G_{n m}^{r}\left(t, t_{1}\right) t_{k \alpha m}^{*} g_{k \alpha}^{<}\left(t_{1}, t^{\prime}\right)\right.\\
&\left.+G_{n m}^{<}\left(t, t_{1}\right) t_{k \alpha m}^{*} g_{k \alpha}^{a}\left(t_{1}, t^{\prime}\right)\right].
\end{aligned}
\end{equation}
This gives the term in current
\begin{equation}
\begin{array}{l}
\sum_{kn} t_{k \alpha n} G_{n, k \alpha}^{<}\left(t, t^{\prime}\right)=\sum_{k mn} \int d t_{1} t_{k \alpha n} t_{k \alpha m}^{*} \\
\times\left[G_{n m}^{r}\left(t, t_{1}\right) g_{k \alpha}^{<}\left(t_{1}, t^{\prime}\right)+G_{n m}^{<}\left(t, t_{1}\right) g_{k \alpha}^{a}\left(t_{1}, t^{\prime}\right)\right] \\
=\sum_{n}\int d t_{1}\left[G^{r}\left(t, t_{1}\right) \Sigma_{\alpha}^{<}\left(t_{1}, t^{\prime}\right)+G^{<}\left(t, t_{1}\right) \Sigma_{\alpha}^{a}\left(t_{1}, t^{\prime}\right)\right]_{n n}
\end{array}
\label{eq:term1}
\end{equation}
matrix element of the self-energy $\Sigma_{\alpha}$ due to lead $\alpha$ is
\begin{equation}
\Sigma_{\alpha,mn}^{\gamma}(t_{1}, t_{2}) = \sum_{k} t_{k\alpha m}^{*}(t_{1}) g_{k\alpha}^{\gamma}(t_{1}, t_{2}) t_{k\alpha n}(t_{2}).
\end{equation}
Here, the matrix index are $m, n$, which is index for energy level of central scattering area. Substitute ~\ref{eq:term} in charge current, we have
\begin{equation}
\begin{aligned}
I_{\alpha}(t) &=-\frac{e}{\hbar} \int d t_{1} \operatorname{Tr}\left[G^{r}\left(t, t_{1}\right) \Sigma_{\alpha}^{<}\left(t_{1}, t\right)\right.\\
&\left.+G^{<}\left(t, t_{1}\right) \Sigma_{\alpha}^{a}\left(t_{1}, t\right)\right]+h . c .
\end{aligned}
\end{equation}
where the summation over index $n$ is abbreviated in to matrix summation notation Tr, and summation index $k$ goes into self-energy matrix $\Sigma_{\alpha}$.
\subsection{Free propagators}
Here we assume a time-dependent external voltage $v_{\alpha}$. The free Green's functions of lead electrons are (XXX)
\begin{equation}
g_{k \sigma}^{<}\left(t, t^{\prime}\right) \equiv i\left\langle c_{k \sigma}^{\dagger}\left(t^{\prime}\right) c_{k \sigma}(t)\right\rangle=i f(\varepsilon_{k}^{(0)}) e^{-i \int_{t'}^{t}dt_{1}\varepsilon_{k\sigma}(t_{1})}
\end{equation}
\begin{equation}
g_{k \sigma}^{>}\left(t, t^{\prime}\right) \equiv-i\left\langle c_{k \sigma}(t) c_{k \sigma}^{\dagger}\left(t^{\prime}\right)\right\rangle=i\left[f\left(\varepsilon_{k}\right)-1\right] e^{-i \varepsilon_{k\sigma}\left(t-t^{\prime}\right)}
\end{equation}
\begin{equation}
g_{k \sigma}^{r}(t) \equiv-i \theta(t)\left\langle\left[c_{k \sigma}(t), c_{k \sigma}^{\dagger}\left(t^{\prime}\right)\right]_{+}\right\rangle=-i \theta(t) e^{-i \varepsilon_{k\sigma}\left(t-t^{\prime}\right)}
\end{equation}
\begin{equation}
g_{k \sigma}^{a}(t) \equiv i \theta(-t)\left\langle\left[c_{k \sigma}(t), c_{k \sigma}^{\dagger}\left(t^{\prime}\right)\right]_{+}\right\rangle=i \theta(-t) e^{-i \varepsilon_{k\sigma}\left(t-t^{\prime}\right)}
\end{equation}
Using the relation
\begin{equation}
\int d t e^{i \omega t}=2 \pi \delta(\omega),
\end{equation}
Fourier transformation gives
\begin{equation}
g_{k \sigma}^{<}(\omega)=2 \pi i f\left(\varepsilon_{k\sigma}\right) \delta\left(\omega-\varepsilon_{k\sigma}\right) = if \left( \varepsilon_{k\sigma} \right) A_{0}(k,\omega)
\end{equation}
\begin{equation}
g_{k \sigma}^{>}(\omega)=2 \pi i\left[f\left(\varepsilon_{k\sigma}\right)-1\right] \delta\left(\omega-\varepsilon_{k\sigma}\right)
\end{equation}
\begin{equation}
g_{k \sigma}^{r}(\omega)= -i \int_{-\infty}^{\infty} d t e^{i \omega t} \theta(t) e^{-i \epsilon_{k\sigma} t}=-i \int_{0}^{\infty} d t e^{i\left(\omega-\epsilon_{k\sigma}\right) t} = \frac{-i}{i(\omega-\varepsilon_{k\sigma})}e^{i(\omega-\varepsilon)}|_{0}^{+\infty}
\end{equation}
To make the integral converge at the upper limit, we let $\omega\rightarrow \omega+i0^{+}$, where $0^{_{+}}$ is a positive infinitesimal, which yields
\begin{equation}
g_{k \sigma}^{r}(\omega) = \frac{1}{\omega-\varepsilon_{k\sigma}+i 0^{+}}.
\end{equation}
Similarly,
\begin{equation}
g_{k \sigma}^{a}(\omega)=\frac{1}{\omega-\varepsilon_{k\sigma}-i 0^{+}}.
\end{equation}
Then we have
\begin{equation}
g_{k \sigma}^{r}(\omega) - g_{k \sigma}^{a}(\omega) = -2\pi i\delta(\omega - \varepsilon_{k\sigma})
\label{eq:r-a}
\end{equation}
The fermion spectral function is defined as
\begin{equation}
\begin{aligned}
A_{0}(k \sigma, \omega) &=i\left[g_{k \sigma}^{r}(\omega)-g_{k \sigma}^{a}( \omega)\right] \\
&=-2 \Im \mathrm{m}\left[g_{k \sigma}^{r}(\omega)\right] \\
&=2 \pi \delta\left(\omega-\varepsilon_{k\sigma}\right)
\end{aligned}
\end{equation}
where the following relation are used
\begin{equation}
\frac{1}{x\pm i \eta}=\mathcal{P} \frac{1}{x}\mp i \pi \delta(x), \quad \eta=0^{+},
\end{equation}
\begin{equation}
\Im \mathrm{m}\left[g_{k \sigma}^{r}(\omega)\right] = -\pi\delta(\omega-\varepsilon_{k}).
\end{equation}
\subsection{DC case}
\begin{equation}
G^{\gamma}\left(t, t_{1}\right)=G^{\gamma}\left(t-t_{1}\right)
\end{equation}
and
\begin{equation}
\Sigma^{\gamma}\left(t, t_{1}\right)=\Sigma^{\gamma}\left(t-t_{1}\right)
\end{equation}
where
\begin{equation}
\gamma=<,>, r, a.
\end{equation}
Recall that
\begin{equation}
\left[G^{<}\right]^{\dagger}(E)=-G^{<}(E)
\end{equation}
\begin{equation}
\left[G^{r}\right]^{\dagger}=G^{a}
\end{equation}
and using equation (221) in WangJian's note, we have charge current for DC bias
\begin{equation}
\begin{aligned}
I_{\alpha} &=-\frac{e}{\hbar} \int \frac{d E}{2 \pi} \operatorname{Tr}\left[\left(G^{r}(E)-G^{a}(E)\right) \Sigma_{\alpha}^{<}(E)\right.\\
&\left.+G^{<}(E)\left(\Sigma_{\alpha}^{a}(E)-\Sigma_{\alpha}^{r}(E)\right)\right]
\end{aligned}
\label{eq:current1}
\end{equation}
Substitute free propagators in, we have
\begin{equation}
\Sigma_{\alpha,mn}^{<}(t-t_{1}) = \sum_{k} t_{k\alpha m}^{*}(t_{1}) g_{k\alpha}^{<}(t_{1}-t_{2}) t_{k\alpha n}(t_{2}) = i\sum_{k} t_{k\alpha m}^{*}(t_{1}) f(\epsilon_{k\alpha})e^{-i\varepsilon_{k\alpha(t-t_{1})}}t_{k\alpha n}(t_{2})
\end{equation}
Fourier transformation gives (dependent variable $\epsilon_{k\alpha}$ not $\omega$?, check Eq.(71) in WangJ's note Chap2?)
\begin{equation}
\Sigma_{\alpha,mn}^{<}(E) = 2\pi i\sum_{k}t_{k\alpha m}^{*}f(\varepsilon_{k\alpha})t_{k\alpha n} \delta(E-\varepsilon_{k\alpha})
\end{equation}
\begin{equation}
\Sigma_{\alpha}^{a}(E)-\Sigma_{\alpha}^{r}(E) = \sum_{k}t_{k\alpha m}^{*}(g_{k\alpha}^{a}(E)-g_{k\alpha}^{r}(E))t_{k\alpha n}
\end{equation}
which according to Eq. (\ref{eq:r-a}), we have
\begin{equation}
\Sigma_{\alpha}^{a}(E)-\Sigma_{\alpha}^{r}(E) = 2\pi i\sum_{k}t_{k\alpha m}^{*}\delta(E-\epsilon_{k\alpha})t_{k\alpha n}.
\end{equation}
Define a level-width function:
\begin{equation}
\Gamma_{\alpha,mn}(E)=\sum_{k} 2 \pi  t_{k\alpha m}^{*}t_{k\alpha n} \delta\left(E-\varepsilon_{k\alpha}\right)
\end{equation}
So it gives equations(the fermion distribution is factorized out of summation $k$?)
\begin{equation}
\Sigma_{\alpha,mn}^{<}(E) = if(\varepsilon_{k\alpha})\Gamma_{\alpha,mn}(E)
\end{equation}
and
\begin{equation}
\Sigma_{\alpha}^{a}(E)-\Sigma_{\alpha}^{r}(E) = i\Gamma_{\alpha,mn}(E)
\end{equation}
Then Eq. (\ref{eq:current1}) can be written as
\begin{equation}
\begin{aligned}
I_{\alpha} =&-\frac{e}{\hbar} \int \frac{d E}{2 \pi} \operatorname{Tr}[\left(G^{r}(E)-G^{a}(E)\right) (if(\varepsilon_{k\alpha}\Gamma_{\alpha,mn}(E)))\\
&+G^{<}(E)(i\Gamma_{\alpha,mn}(E))]\\
=& -\frac{ie}{\hbar} \int \frac{d E}{2 \pi} \operatorname{Tr}\left[\Gamma_{\alpha,mn}(E)(\left[G^{r}(E)-G^{a}(E)\right] f(\varepsilon_{k\alpha}) +G^{<}(E))\right]
\end{aligned}
\label{eq:dc-current}
\end{equation}
In steady state, $I=I_{L}=-I_{R}$, or $I=I_{L}+I_{R}=(I_{L}-I_{R})/2$, this leads to the general expression for the dc-current
\begin{equation}
\begin{aligned}
I=& -\frac{\mathrm{i} e}{2 \hbar} \int \frac{\mathrm{d} \varepsilon}{2 \pi} \operatorname{Tr}\left\{\left[\boldsymbol{\Gamma}^{L}(\varepsilon)-\boldsymbol{\Gamma}^{R}(\varepsilon)\right] \mathbf{G}^{<}(\varepsilon)\right.\\
&\left.+\left[f_{L}(\varepsilon) \boldsymbol{\Gamma}^{L}(\varepsilon)-f_{R}(\varepsilon) \boldsymbol{\Gamma}^{R}(\varepsilon)\right]\left[\mathbf{G}^{\mathrm{r}}(\varepsilon)-\mathbf{G}^{\mathrm{a}}(\varepsilon)\right]\right\}
\end{aligned}
\end{equation}
if the left and right line-width functions are proportional to each other,
\begin{equation}
\boldsymbol{\Gamma}^{L}(\varepsilon)=\lambda \boldsymbol{\Gamma}^{R}(\varepsilon)
\end{equation}
and fix the arbitrary parameter $x$, i.e. $x=1 /(1+\lambda)$, gives
\begin{equation}
\begin{aligned}
J &=\frac{1 e}{\hbar} \int \frac{\mathrm{d} \varepsilon}{2 \pi}\left[f_{L}(\varepsilon)-f_{R}(\varepsilon)\right] \mathcal{T}(\varepsilon) \\
\mathcal{T}(\varepsilon) &=\operatorname{Tr}\left\{\frac{\Gamma^{L}(\varepsilon) \Gamma^{R}(\varepsilon)}{\Gamma^{L}(\varepsilon)+\Gamma^{R}(\varepsilon)}\left[\mathbf{G}^{\mathrm{r}}(\varepsilon)-\mathbf{G}^{\mathrm{a}}(\varepsilon)\right]\right\}
\end{aligned}
\end{equation}
Despite the apparent similarity of (12.27) to the Landauer formula, it is important to bear in mind that, in general, there is no immediate connection between the quantity $\mathcal{T}(\varepsilon)$ and the transmission coefficient $T(\varepsilon)$.
\subsection{Another way to get $G_{n,k\alpha}(\tau,\tau')$(Dyson equation + Keldysh equation)}
Denote $G_{0}$ the Green’s function of the isolated quantum dot and leads corresponding to the Hamiltonian $H_{0}$, and $G$ the Green’s function of the open system corresponding to $H$, one has the Dyson equation
\begin{equation}
G=G_{0}+G_{0} \Sigma G
\end{equation}
Use the theorem of analytic continuation on Dyson equation, we get the Keldysh equation (\textcolor{red}{in matrix representation})
\begin{equation}
G^{<,>}=G_{0}^{<,>}+G^{r} \Sigma^{r} G_{0}^{<,>}+G^{<,>} \Sigma^{a} G_{0}^{a}+G^{r} \Sigma^{<,>} G_{0}^{a}
\end{equation}
or
\begin{equation}
G^{<,>}=G_{0}^{<,>}+G_{0}^{r} \Sigma^{r} G^{<,>}+G_{0}^{<,>} \Sigma^{a} G^{a}+G_{0}^{r} \Sigma^{<,>} G^{a}
\end{equation}
or
\begin{equation}
G^{<}=G^{r}\left(G_{0}^{r}\right)^{-1} G_{0}^{<}\left(G_{0}^{a}\right)^{-1} G^{a}+G^{r} \Sigma^{<} G^{a}
\end{equation}
See Eq. (77) in WangJian's notes.
\subsection{With spin index}
The demonstrative current of lead $\beta$ with spin $\sigma$ is~\cite{CaoZhan}
\begin{equation}
\begin{array}{r}
I_{\beta \sigma}=\frac{e}{h} \sum_{k, i, j} \int d \omega V_{k i \beta \sigma} V_{k j \beta \sigma}^{*}\left\{\left[G_{i \sigma, j \sigma}^{r}(\omega)-G_{i \sigma, j \sigma}^{a}(\omega)\right] g_{k \beta \sigma}^{<}(\omega)\right. \\
\left.-\left[g_{k \beta \sigma}^{r}(\omega)-g_{k \beta \sigma}^{a}(\omega)\right] G_{i \sigma, j \sigma}^{<}(\omega)\right\}.
\end{array}
\end{equation}
Substitute free propagators into current formula, we have
\begin{equation}
I_{\beta \sigma}=\frac{i e}{h} \sum_{i, j} \int d \omega \Gamma_{i j \beta \sigma}(\omega)\left\{\left[G_{i \sigma, j \sigma}^{r}(\omega)-G_{i \sigma, j \sigma}^{a}(\omega)\right] f_{\beta}(\omega)+G_{i \sigma, j \sigma}^{<}(\omega)\right\}
\end{equation}
self-energy of lead $\alpha$ is
\begin{equation}
\Sigma_{\alpha}^{<}(\omega)=i \Gamma_{\alpha}\left(\omega-q v_{\alpha}\right) f_{\alpha}(\omega)
\end{equation}


\clearpage
\section{Appendix: Analytic Continuation}

We list here all the analytic continuations used in this work.  For $C = AB$ (matrix multiplication), we have\cite{jauho}
\begin{gather}
C^< = A^r B^< + A^< B^a, ~~ {\rm and } ~~ C^> = A^r B^> + A^> B^a \label{conti1}
\end{gather}
and
\begin{gather}
C^r = A^r B^r , ~~ {\rm and } ~~ C^a =  A^a B^a \label{conti2}
\end{gather}
For $C(\tau,\tau') = A(\tau,\tau') B(\tau,\tau')$ or $C=A.B$ (the Hadamard matrix product), we have\cite{jauho}
\begin{gather}
C^< = A^<. B^< , ~~ {\rm and } ~~ C^> = A^> .B^> \label{conti3}
\end{gather}
and
\begin{eqnarray}
&&C^r = A^r .B^< + A^< .B^r + A^r .B^r, \nonumber \\
&& C^a = A^a .B^< + A^< .B^a + A^a. B^a \label{conti4}
\end{eqnarray}
For $C(\tau,\tau') = A(\tau,\tau') B(\tau',\tau)$ or $C = A.{\tilde B}$ where ${\tilde B}(t_1,t_2) \equiv B(t_2,t_1)$, we have\cite{jauho}
\begin{gather}
C^< = A^< .{\tilde B}^> , ~~ {\rm and } ~~ C^> = A^> .{\tilde B}^< \label{conti5}
\end{gather}
\begin{eqnarray}
&& C^r = A^< .{\tilde B}^a + A^r .{\tilde B}^< , \nonumber \\
&& C^a = A^< .{\tilde B}^r + A^a .{\tilde B}^<  \label{conti6}
\end{eqnarray}
For $D=ABC$, we have
\begin{equation}
D^{<}=A^r B^r C^{<}+A^r B^{<} C^a+A^{<} B^a C^a,
\end{equation}
and 
\begin{equation}
D^r=A^r B^r C^r.
\end{equation}
From Eqs.~(\ref{conti5}) and (\ref{conti6}), one can easily check the relation $C^> - C^< = C^r - C^a$ which must be satisfied.


\begin{thebibliography}{10}
\bibitem{ref1}
Y, K, Kato. Observation of the Spin Hall Effect in Semiconductors[J]. Science, 2004.
\bibitem{Jauho}
Antti-Pekka Jauho, Quantum Kinetics in Transport and Optics of Semiconductors, P188.
\bibitem{Wu} L. Gu, H. H. Fu, and R. Q. Wu, Phys. Rev. B 94, 115433 (2016).
\bibitem{Ren} J. Ren, Phys. Rev. B 88, 220406 (2013).
\end{thebibliography}


% \end{CJK}
\end{document}


